\chapter*{Запуск экземпляров программ}
\markboth{\MakeUppercase{Запуск экземпляров программ}}{}
\addcontentsline{toc}{chapter}{Запуск экземпляров программ}



Книга содержит много примеров программ и программных фрагментов. Все из них могут быть запущены на Системе Программирования Моцарт. Для упрощения выполнения программ, пожалуйста, помните о следующих пунктах:

\begin{itemize}
\item{Систему Моцарт можно безвозмездно скачать с Web сайта Консорциума Моцарт \verb|http://www.mozart-oz.org|. Существуют выпуски для различных версий Windows, Unix и Mac OS X.}

\item{Все примеры, за исключением некоторых отдельных приложений, можно запускать в интерактивной среде разработки Моцарт. Приложение A содержит введение в эту среду.}

\item{Новые переменные в интерактивных примерах должны объявляться с помощью оператора \lstinline!declare!. Примеры в Главе 1 показывают как производить это действие. Если вы забудете выполнить это действие, то в результате можете получить различные странные ошибки, в случае если уже существуют старые версии переменных. Начиная с Главы 2 и в последующих главах, оператор \lstinline!declare! будет опускаться в тех случаях, когда очевидно присутствие новых переменных. В этом случае для запуска примеров нужно добавлять определение новой переменной.}

\item{Некоторые главы используют операции, не являющиеся частью стандартного выпуска Моцарт. Исходный код этих дополнительных операций (наряду с множеством других полезных материалов) содержится на Web сайте книги. Мы рекомендуем вставить эти определения в ваш \verb|.ozrc| файл, таким образом они будут автоматически загружены при запуске системы.}

\item{Существует несколько различий между идеальной реализацией, применяемой в этой книге и системой Моцарт. Эти различия описаны на Web сайте книги.}
\end{itemize}
